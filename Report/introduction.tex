Meep is a free and open-source software package for the simulation of electromagnetic fields via the finite-difference time-domain (FDTD) method.
Its application extend also to the simulation of photonic elements and circuits.  

The finite-difference time-domain (FDTD) method is a widely used technique: the space is divided into a discrete grid and the fields are evolved in discrete time steps.
The discretization of time id directly determined by the the space discretization and as the steps are made finer and finer, the simulation becomes a closer and closer approximation for the true continuous equations.

Considering the 1D case for simplicity, the FDTD algorithm starts from the Ampere's and Faraday's laws written in Eq. \ref*{eq:Ampere_Faraday}.
\begin{equation} \label{eq:Ampere_Faraday}
    \mu \pdv{H_y}{t} = \pdv{E_z}{x} \qquad\qquad\qquad \varepsilon \pdv{H_z}{t} = \pdv{H_y}{x}
\end{equation}
As the space and time are discretized, the electric and magnetic fields can be written using a descrete notation as \(E_z(x, t) = E_z (m\Delta x, q \Delta t) = E_z^q[m]\) and \(H_y(x, t) = H_y(m\Delta x, q \Delta t) = H_y^q[m]\). This allows to replace the derivatives in the laws with finite differences, obtaining the difference equation in Eq. \ref*{eq:difference_eqs}.
\begin{align}\label{eq:difference_eqs}
    \mu \frac{H_y^{q+\frac{1}{2}}[m+\frac{1}{2}] - H_y^{q-\frac{1}{2}}[m+\frac{1}{2}]}{\Delta t} = \frac{E_z^q[m+1] - E_z^q[m]}{\Delta x}
    \\
    \varepsilon\frac{E_z^{q+1}[m] - E_z^q[m]}{\Delta t} = \frac{H_y^{q+\frac{1}{2}}[m+\frac{1}{2}] - H_y^{q+\frac{1}{2}}[m-\frac{1}{2}]}{\Delta x}
\end{align}
Finally, these difference equations are turned into update equation, written in Eq. \ref*{eq:update_eqs}, to determine the ''future fields'' knowing the ''past ones''
\begin{align}\label{eq:update_eqs}
    H_y^{q+\frac{1}{2}} \left[ m+\frac{1}{2} \right] = H_y^{q-\frac{1}{2}} \left[ m+\frac{1}{2} \right] + \frac{\Delta t}{\mu \Delta x} \left( E_Z^q[m+1] - E_z^q[m] \right)
    \\
    E_z^{q+1} [m] = E_z^q[m] + \frac{\Delta t}{\varepsilon \Delta x} \left( H_y^{q+\frac{1}{2}}\left[ m+\frac{1}{2} \right] - H_y^{q+\frac{1}{2}}\left[ m-\frac{1}{2} \right] \right)
\end{align}

\subsection{Simulation}